%--------------------
% Header
%--------------------
\documentclass[11pt]{article}
\usepackage[margin=0.75in]{geometry}
\usepackage{enumitem}
\usepackage[hidelinks]{hyperref}

\hypersetup{
    pdftitle={Tom McKernan - Software Engineer Resume},
    pdfauthor={Tom McKernan},
    pdfkeywords={Python, Rust, Cloud, DevOps, Microservices, AWS, Kubernetes, CI/CD, Terraform, GitHub Actions, Jenkins, Docker}
}

% Formatting
\renewcommand{\familydefault}{\sfdefault}
\setlist[itemize]{noitemsep, topsep=0pt}
\pagestyle{empty}

\begin{document}

%--------------------
% Header
%--------------------
    \begin{center}
    {\LARGE \textbf{Thomas McKernan}}
        \\
        \vspace{2mm}
        \href{mailto:tmckernan36@gmail.com}{tmckernan36@gmail.com} \quad|\quad
        \href{https://github.com/mckernant1}{github.com/mckernant1}
    \end{center}

%--------------------
% Summary
%--------------------
    \section*{Professional Summary}
    Software Engineer with 5+ years of experience in cloud infrastructure, DevOps automation, and distributed systems.
    Skilled in Java, Kotlin, Python, and Rust, with expertise in CI/CD pipelines, Kubernetes, Docker, Terraform, AWS, and cloud-native solutions.
    Proven track record of optimizing AWS costs, improving system performance, troubleshooting customer issues, and mentoring team members to enhance engineering practices.

%--------------------
% Skills
%--------------------
    \section*{Skills}
    \textbf{Programming:} Java, Kotlin, Python, Rust, Bash \\
    \textbf{Cloud \& DevOps:} AWS (S3, DynamoDB, EKS, ECR, SQS), Kubernetes, Docker, Helm, CloudFormation, GitHub Actions, Jenkins, CI/CD, Terraform \\
    \textbf{Tools \& Frameworks:} Spring, Linux, Typescript, Git \\
    \textbf{Other:} Agile, Scrum, Mentoring, Problem-Solving, Ownership

%--------------------
% Experience
%--------------------
    \section*{Experience}

    \noindent\textbf{Senior Software Engineer} \hfill 08/2023 -- Present \\
    SS\&C Advent
    \begin{itemize}
        \item Migrated company CI systems and workflows from Jenkins to modern cloud-native pipelines with GitHub Actions, improving build reliability.
        \item Built custom automation in TypeScript to standardize component builds across teams.
        \item Managed Kubernetes deployments using Helm, reducing operational friction and improving rollout safety.
        \item Maintained infrastructure-as-code workflows with Terraform and CloudFormation, improving repeatability of cluster configuration.
        \item Maintained development environments, ensuring stability and reliability for all engineering teams.
        \item Added monitoring and alerting to development environments, proactively detecting issues and reducing downtime.
        \item Mentored engineers on CI/CD practices, pipeline design, and infrastructure-as-code workflows.
        \item Collaborated with customer support and account teams to troubleshoot and resolve critical client issues, improving satisfaction and reducing production impact.
        \item Partnered with development and QA teams to ensure seamless release coordination and reduce rollback incidents.
    \end{itemize}

    \noindent\textbf{Software Engineer II} \hfill 03/2022 -- 07/2023 \\
    Amazon Web Services
    \begin{itemize}
        \item Analyzed and optimized AWS S3 usage patterns, reducing monthly storage costs by 60\%.
        \item Designed high-performance Java data structures to improve CPU efficiency by 20\%.
        \item Partnered with cross-team stakeholders to deliver features that improved real-time availability metrics.
        \item Automated global parity and scalability through improved CI/CD processes with Amazon's Internal Tooling.
        \item Implemented a TTL system with a distributed locking mechanism for scanning and expiring database items, improving data consistency and reliability.
        \item Developed internal tooling in Python and Java for collecting logs and internal metrics, streamlining debugging and operational workflows.
        \item Served as on-call engineer for production systems, responding to incidents, troubleshooting issues, and ensuring high availability and reliability for customers.
        \item Collaborated on a Rust storage engine and performance testing, working with senior engineers to improve system throughput and scalability.
    \end{itemize}

    \noindent\textbf{Software Engineer I} \hfill 02/2020 -- 03/2022 \\
    Amazon Web Services
    \begin{itemize}
        \item Upgraded Python systems from 2.7 to 3.x, preventing EOL issues and security vulnerabilities.
        \item Built and launched CloudWatch Contributor Insights dashboards for DynamoDB using CloudFormation.
        \item Investigated and resolved persistent DynamoDB throttling using logs and metrics analysis.
        \item Implemented automation for ticket data enrichment using custom Python tooling.
        \item Served as on-call engineer for production systems, responding to incidents, troubleshooting issues, and ensuring high availability and reliability for customers.
    \end{itemize}

    \noindent\textbf{Software Engineer Intern - Back End Team} \hfill 05/2019 -- 08/2019 \\
    Peapod Digital Labs
    \begin{itemize}
        \item Built and maintained a sidecar service supporting Peapod microservices, improving modularity and deployment flexibility.
        \item Researched Docker and Kubernetes to support migration from monolithic architecture to microservices.
        \item Conducted performance testing using Gatling and Vegeta to identify bottlenecks and optimize service latency.
        \item Developed internal full-stack applications to automate CRUD operations and improve team efficiency.
    \end{itemize}

    \vspace{2mm}

    \noindent\textbf{Software Engineer Intern - Front End Team} \hfill 06/2018 -- 03/2019 \\
    Peapod Digital Labs
    \begin{itemize}
        \item Developed a reusable Vue.js component library to standardize UI elements across web applications.
        \item Researched integration strategies for Vue and Angular frameworks to support hybrid projects.
        \item Investigated cross-platform frameworks including NativeScript, Xamarin Forms, and Flutter for mobile app development.
        \item Collaborated with the front-end team to implement unit tests, ensuring component reliability and maintainability.
    \end{itemize}


    \section*{Projects}

    \noindent\textbf{League of Legends Esports Platform} — Discord Bot, API Service, Data Loader, and OpenAPI Specification
    \begin{itemize}
        \item Designed and built an end-to-end esports data platform consisting of a Kotlin-based Discord bot, a backend API service, and a formal OpenAPI specification.
        \item Developed a feature-rich Discord bot providing match schedules, results, standings, user predictions, and personalized settings for global esports regions.
        \item Implemented a backend API service responsible for data aggregation, normalization, caching, and persistence, decoupling Discord interactions from business logic.
        \item Authored a comprehensive OpenAPI specification to standardize service contracts, enable client generation, and improve maintainability.
        \item Integrated with third-party esports data providers and handled region-specific formats, time zones, and tournament structures.
        \item Designed persistent storage and stateful workflows to support concurrent users and prediction tracking.
    \end{itemize}

    \vspace{4mm}

    \noindent\textbf{Infrastructure Libraries \& Platform Tooling} — Kotlin Utilities, Metrics, and Public Infrastructure
    \begin{itemize}
        \item Developed and maintained shared Kotlin utility libraries to standardize patterns across services, including configuration handling, logging, error modeling, and concurrency primitives.
        \item Built a lightweight metrics framework to simplify instrumentation and expose consistent telemetry for monitoring and alerting.
        \item Designed metrics abstractions to support multiple backends while minimizing coupling.
        \item Created and managed public infrastructure repositories defining reusable CI/CD workflows, infrastructure templates, and operational conventions.
        \item Codified infrastructure and release practices to improve consistency, reliability, and onboarding across projects.
        \item Emphasized strong typing, composability, and clear APIs to reduce duplication and operational risk in distributed systems.
        \item Designed AWS CDK-based deployment pipelines to automate the release of shared libraries and infrastructure components.
        \item Integrated GitHub-native security tooling to enforce dependency updates and prevent vulnerable artifacts from being released.
    \end{itemize}

    \vspace{4mm}

    \noindent\textbf{Kotlin/Spring gRPC}
    \begin{itemize}
        \item Improved gRPC error handling in the grpc-spring library by implementing proper exception translation and mapping between \texttt{ResponseStatusException} and gRPC \texttt{Status} codes.
        \item Contributed to fixing \texttt{SecurityContextHolder} for kotlin coroutine based grpc services.
    \end{itemize}


%--------------------
% Education
%--------------------
    \section*{Education}
    \textbf{University of Dayton} \hfill 08/2016 -- 12/2019 \\
    B.S.\ Computer Science, Minor in Mathematics

\end{document}
